

\section{Công đoạn chung của phương pháp học sâu trong bài toán sinh cử chỉ}
\label{sec:commonstage}

Như đã trình bày ở \autoref{sec:Data}, cử chỉ bao gồm chuỗi chuyển động của toạ độ điểm 3D. Với mỗi tập dữ liệu số lượng xương (bone) mỗi khung hình sẽ khác nhau. 

Cách tiếp cận bằng học sâu với bài toán sinh cử chỉ (gesture generation) được thực hiện với nhiều phương pháp khác nhau. Tuy nhiên luận văn tổng quát hoá lại thành các công đoạn chính như \autoref{fig:CommonStage} sau:

\begin{figure}[H]
	\centering
	\includegraphics[width=\textwidth]{CommonStage}
	\caption{Các công đoạn trong mô hình sinh cử chỉ cử chỉ.}
	\label{fig:CommonStage}
\end{figure}

\begin{enumerate}[label=\textbf{\arabic*.}]
	\item \textbf{Tiền xử lý dữ liệu}: Trong công đoạn tiền xử lý, các dữ liệu về giọng nói ở dạng định wav, tệp BVH và văn bản sẽ được đọc, số hoá để thu được các vector hoặc ma trận thể hiện thông tin thô của dữ liệu. Đối với mỗi phương pháp học khác nhau, các thông tin dữ liệu ban đầu sẽ được chọn để học cũng khác nhau.
	
	\item \textbf{Xử lý đặc trưng}: Trong công đoạn xử lý đặc trưng, các dữ liệu thô như giọng nói và văn bản được nhúng (embedding) để biểu diễn thành các vector đặc trưng. Các phương pháp khác nhau sẽ chọn các mô hình nhúng khác nhau. Việc biểu diễn các cử chỉ của nhân vật thành các vector đặc trưng của mỗi phương pháp cũng khác nhau.
	
	\item \textbf{Trích xuất đặc trưng}: Công đoạn trích xuất đặc trưng sẽ dùng các lớp biến đổi tuyến tính (linear) hoặc các lớp CNN để trích xuất các đặc trưng của dữ liệu. Các dữ liệu về văn bản hoặc giọng nói sau khi xử lý đặc trưng có thể cũng được cho đi qua các lớp trích xuất đặc trưng để biểu diễn thành các vector đặc trưng để biểu diễn tương ứng với văn bản và giọng nói.
	
	\item \textbf{Mã hoá đặc trưng}: Trong công đoạn mã hoá đặc trưng, các vector về cử chỉ, cảm xúc, và giọng nói sẽ được biểu diễn lên không gian tiềm ẩn nhỏ hơn kích thước ban đầu nhằm thuận tiện cho việc tính sự tương quan giữa các đặc trưng ở công đoạn kết hợp đặc trưng.
	
	\item \textbf{Kết hợp đặc trưng}: Trong công đoạn kết hợp đặc trưng, các đặc trưng giọng nói, văn bản, cử chỉ cũng như các thông tin khác được kết hợp với nhau bằng việc concat, các lớp kết nối đầy đủ hoặc kết hợp các đặc trưng bằng cách cộng hoặc trừ các vector tiềm ẩn.
	
	\item \textbf{Giải mã đặc trưng}: Trong công đoạn giải mã đặc trưng, các đặc vector tiềm ẩn sẽ được giải mã hay tăng chiều dữ liệu về kích thước ban đầu.
	
	\item \textbf{Kết xuất} (Render): Sau khi có được vector ở kích thước ban đầu, các vector sẽ được biến đổi ngược trở về các tệp BVH để kết xuất bằng các phần mềm như Blender hoặc Unity để minh học các chuyển động của nhân vật.
\end{enumerate}



\section{Diffusion-base Model}
\label{sec:diffusionbase}



Với đặc điểm dữ liệu là giá trị của các góc quay, các toạ độ của điểm khớp, nên cần độ chi tiết cao để tạo ra sự chân thực trong các chuyển động của nhân vật. Ngoài ra dữ liệu sẽ thiếu và rất ít dữ liệu trong các trường hợp cực trị của tham số.
Nên luận văn sử dụng mô hình Diffusion, với đặc điểm là có thể học được độ chi tiết cao hơn và có thể phủ được dữ liệu trong các trường hợp cực trị của tham số và độ phủ về mật độ dữ liệu thấp.

