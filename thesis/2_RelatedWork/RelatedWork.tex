\chapter{TỔNG QUAN}
\label{Chapter2}

Bài toán sinh cử chỉ cũng tương tự như các bài toán khác đều đã nghiên cứu và phát triển song hành với các phương pháp học máy truyền thống và hiện đại. Gồm các nhóm phương pháp dựa trên luật và các phương pháp dựa trên dữ liệu.  Đầu tiên luận văn chứng minh mối quan hệ giữa cử chỉ và giọng nói \autoref{sec:relationspeechandgesture}, từ đó là tiền đề thể hiện sự đồng bộ về dữ liệu giữa cử chỉ và giọng nói và việc học mối quan hệ giữa cử chỉ và giọng nói có ý nghĩa. Trong mục \autoref{sec:commonstage} Luận văn sẽ trình bày về các công đoạn chung trong các phương pháp sinh cử chỉ.
Ở phần \autoref{sec:relatedwork} luận văn trình bày về các phương pháp đã được sử dụng trong quá trình sinh cử chỉ. Luận văn sẽ so sánh các phương pháp, từ đó nêu lý do luận văn sử dụng mô hình diffusion để áp dụng cho bài toán sinh cử chỉ.
Trong phần \autoref{sec:diffusionbase}, luận văn sẽ trình bày về cách các mô hình diffusion được áp dụng cho bài toán sinh cử chỉ gần đây.

\section{Mối quan hệ giữa cử chỉ và giọng nói}
\label{sec:relationspeechandgesture}

Theo ngôn ngữ học, cử chỉ có thể được phân thành 6 nhóm chính: cử chỉ thích nghi (adaptors), cử chỉ biểu tượng (emblems), cử chỉ chỉ định (deictics), cử chỉ biểu trưng (iconics), cử chỉ ẩn dụ (metaphorics), và cử chỉ nhấn mạnh (beat) \cite{ekman1969repertoire}, \cite{sebeok2011advances}. Trong số đó, cử chỉ nhấn mạnh không mang ý nghĩa ngữ nghĩa trực tiếp nhưng đóng vai trò quan trọng trong việc đồng bộ nhịp điệu giữa giọng nói và cử chỉ \cite{kipp2005gesture}, \cite{sebeok2011advances}. Tuy nhiên, nhịp điệu giữa giọng nói và cử chỉ nhấn mạnh không hoàn toàn đồng bộ, khiến việc học mối quan hệ thời gian giữa chúng trở nên phức tạp \cite{mcclave1994gestural}, \cite{bhattacharya2021speech2affectivegestures}, \cite{kucherenko2020gesticulator}, \cite{yoon2020speech}.

Cử chỉ tương tác với các cấp độ thông tin khác nhau trong giọng nói \cite{sebeok2011advances}. Chẳng hạn, cử chỉ biểu tượng, như hành động giơ ngón cái, thường liên quan đến thông tin ngữ nghĩa cấp cao (ví dụ: "tốt" hoặc "tuyệt vời"), trong khi cử chỉ nhấn mạnh thường đi kèm với thông tin cấp thấp như nhấn mạnh trong âm thanh. Các nghiên cứu trước đây thường chỉ sử dụng đặc trưng từ lớp cuối cùng của bộ mã hóa giọng nói để tổng hợp cử chỉ \cite{alexanderson2020style}, \cite{bhattacharya2021speech2affectivegestures}, \cite{kucherenko2021large}, \cite{qian2021speech}, \cite{yoon2022genea}. Tuy nhiên, cách tiếp cận này có thể làm trộn lẫn thông tin từ nhiều cấp độ, dẫn đến khó khăn trong việc phân tách rõ ràng nhịp điệu và ngữ nghĩa.

Như các nghiên cứu ngôn ngữ học chỉ ra \cite{kipp2005gesture}, \cite{neff2008gesture}, \cite{webb1997linguistic}, cử chỉ trong giao tiếp hàng ngày có thể được chia thành một số lượng giới hạn các đơn vị ngữ nghĩa với các biến thể chuyển động khác nhau. Dựa trên giả định này, được phân tách đặc trưng giọng nói thành hai loại: đặc trưng cấp cao đại diện cho các đơn vị ngữ nghĩa, và đặc trưng cấp thấp xác định các biến thể chuyển động. Từ đó, mối liên hệ giữa chúng được học thông qua các lớp khác nhau của bộ mã hóa giọng nói. Các thử nghiệm chứng minh rằng cơ chế này có khả năng tách biệt rõ ràng các đặc trưng ở nhiều cấp độ, đồng thời tổng hợp được cử chỉ phù hợp về ngữ nghĩa và phong cách.


\section{Các công đoạn chung trong bài toán sinh cử chỉ}
\label{sec:commonstage}

Cách tiếp cận với bài toán sinh cử chỉ (gesture generation) được thực hiện với nhiều phương pháp khác nhau. Tuy nhiên luận văn tổng quát hoá lại thành các công đoạn chính như \autoref{fig:CommonStage} sau:

\begin{figure}[H]
	\centering
	\includegraphics[width=\textwidth]{CommonStage}
	\caption{Các công đoạn trong mô hình sinh cử chỉ cử chỉ.}
	\label{fig:CommonStage}
\end{figure}

Như đã trình bày ở \autoref{sec:ProblemStatement}, cử chỉ bao gồm chuỗi chuyển động của toạ độ điểm 3D bao gồm 

\begin{enumerate}[label=\textbf{\arabic*.}]
	\item \textbf{Tiền xử lý dữ liệu}: 
	
	\item \textbf{Xử lý đặc trưng}: 
	
	\item \textbf{Trích xuất đặc trưng}
	
	\item \textbf{Mã hoá đặc trưng}
	
	\item \textbf{Kết hợp đặc trưng}
	
	\item \textbf{Giải mã đặc trưng}
	
	\item \textbf{Kết xuất}
\end{enumerate}

\section{Tổng quan các phương pháp cho bài toán sinh cử chỉ}
\label{sec:relatedwork}



\subsection{Phương pháp dựa trên luật}

Các phương pháp dựa trên luật thường ánh xạ (mappings) từng giọng nói với từng đơn vị cử chỉ \cite{huang2012robot}. Và luật được tạo thủ công. Phương pháp dựa trên luật thì chúng ta có thể dễ dàng điều khiển kết quả của mô hình và có khả năng giải thích tốt kết quả dự đoán của mô hình.
Tuy nhiên chi phí để tạo thủ công là không khả thi để xây dựng cho các ứng dụng phức tạp đòi hỏi phải xử lý một lượng dữ liệu rất lớn.

\subsection{Phương pháp dựa trên thống kê}

Tương tự như phương pháp dựa trên luật, phương pháp dựa trên dữ liệu cũng ánh xạ các đặc trưng của giọng nói tương ứng với cử chỉ nhưng thay vì làm thủ công thì được sử dụng học một cách tự động dựa trên dữ liệu.
Trong đó có hai phương pháp chính là phương pháp thống kê và phương pháp dựa trên dữ liệu.

%\subsubsection{Phương pháp thống kê}

Phương pháp thống kê sử dụng phân phối xác suất để tìm sự tương đồng giữa các đặc trưng giọng nói và cử chỉ \cite{levine2010gesture}. Tác giả \cite{neff2008gesture} xây dựng mô hình để học từng phong cách của từng người nói.

\subsection{Phương pháp học sâu}

\setcounter{figure}{3}
\begin{figure}[H]
	\centering
	\includegraphics[width=0.8\textwidth]{GeneralOverview}
	\caption{Tổng quan về các mô hình tạo sinh khác nhau.}
	\label{fig:GeneralOverview}
\end{figure}

Phương pháp sinh cử chỉ được chia thành hai nhóm chính. Bao gồm các mô hình ước lượng log likelihood (likelihood-based model)  và phương pháp dựa vào các mô hình sinh ngầm định (implicit generative models) \cite{song2021score}. 

\subsubsection{Likelihood-based Model}

Phương pháp học ước lượng log likelihood là phương pháp học trực tiếp từ hàm mật độ xác xuất (probability density) thông qua maximum likelihood. Các phương pháp điển hình là autoregressive models, normalizing flow models, energy-based models (EBMs)
, và variational auto-encoders (VAEs).

%Phương pháp học sâu sử dụng mạng nơ-ron (neural) thông qua nhiều lớp ẩn để học một cách tự động các phối xác xuất giữa cử chỉ và giọng nói.

Mô hình được kết hợp với văn bản đầu vào được gắn thẻ với chủ đề, trọng tâm câu và thành ngữ để tạo ra các kịch bản cử chỉ, sau đó được ánh xạ sang một chuỗi các cử chỉ được chọn từ một từ điển hoạt họa. \cite{chiu2015predicting} huấn luyện một model classifier neural network để chọn một đơn vị cử chỉ phù hợp dựa trên đầu vào là giọng nói. 
%Nghiên cứu gần đây đã bắt đầu tận dụng học sâu và huấn luyện các mô hình kết thúc đến cuối sử dụng dữ liệu cử chỉ thô trực tiếp, giải phóng các nỗ lực thủ công trong thiết kế từ điển cử chỉ và các quy tắc ánh xạ.

Cử chỉ có thể được tổng hợp bằng các mô hình xác định như perceptron đa tầng (MLP) \cite{kucherenko2020gesticulator}, recurrent neural networks \cite{bhattacharya2021speech2affectivegestures}, \cite{liu2022learning}, \cite{hasegawa2018evaluation}, \cite{yoon2020speech}, convolutional networks \cite{habibie2021learning} và transformer \cite{bhattacharya2021text2gestures} 

\subsubsection{Implicit Generative Models}

Trong các phương pháp dựa vào mô hình sinh ngầm định, phân phối của dữ liệu được học một cách ngầm định thông qua việc quá trình lấy mẫu (sampling). Ví dụ tiêu biểu nhất là mô hình generative adversarial networks (GANs). Khi dữ liệu được tổng hợp bằng cách chuyển phân phối dữ liệu ban đầu ở dạng phân phối chuẩn về phân phối của dữ liệu.

%
%\begin{table}[h!]
%	\small
%	\centering
%	\renewcommand{\arraystretch}{1.5} % Tăng khoảng cách giữa các hàng
%	\begin{tabular}{|p{0.2\textwidth}|p{0.35\textwidth}|p{0.35\textwidth}|}
%		\hline
%		\textbf{Loại phương pháp} & \textbf{Ưu điểm} & \textbf{Nhược điểm} \\ \hline
%		Rule-Based  & 
%		- Dễ hiểu và dễ triển khai. \newline 
%		- Dễ giải thích và có thể kiểm soát được \newline
%		- Hiệu quả trong các trường hợp đơn giản hoặc dữ liệu nhỏ. & 
%		- Không tổng quát hoá tốt với dữ liệu phức tạp. \newline 
%		- Đòi hỏi nhiều công sức trong việc xây dựng quy tắc thủ công. \\ \hline
%		Likelihood-Based Models & 
%		- Khả năng ước lượng mật độ xác suất của dữ liệu \newline 
%		- Có khả năng mở rộng và học từ dữ liệu lớn & 
%		- Dễ bị ảnh hưởng bởi nhiễu \newline 
%		- Kết quả thấp ở vùng dữ liệu hiếm \newline
%		- Khả năng sinh không đa dạng \\ \hline
%		Implicit Generative Models & 
%		- Tạo ra dữ liệu chất lượng cao. \newline 
%		- Linh hoạt và đa dạng \newline
%		- Phủ được vùng có mật độ dữ liệu thấp & 
%		- Cần cấu hình phức tạp để đạt hiệu năng tốt. \newline 
%		- Khó đánh giá do mỗi lần nhiễu là khác nhau. \newline 
%		- Quá trình lấy mẫu chậm \\ \hline
%	\end{tabular}
%	\caption{Bảng so sánh ưu và nhược điểm của các phương pháp}
%\end{table}

\begin{table}[h!]
	\small
	\centering
	\renewcommand{\arraystretch}{1.5} % Tăng khoảng cách giữa các hàng
	\resizebox{\textwidth}{!}{ % Giảm kích thước bảng để vừa với trang
		\begin{tabular}{|p{0.2\textwidth}|p{0.35\textwidth}|p{0.35\textwidth}|p{0.2\textwidth}|}
			\hline
			\textbf{Phương pháp tiêu biểu} & \textbf{Ưu điểm} & \textbf{Nhược điểm} & \textbf{Loại phương pháp} \\ \hline
			Robot behavior toolkit \cite{huang2012robot} & 
			- Dễ hiểu và dễ triển khai. \newline 
			- Dễ giải thích và có thể kiểm soát được \newline
			- Hiệu quả trong các trường hợp đơn giản hoặc dữ liệu nhỏ. & 
			- Không tổng quát hoá tốt với dữ liệu phức tạp. \newline 
			- Đòi hỏi nhiều công sức trong việc xây dựng quy tắc thủ công. & 
			Rule-Based  \\ \hline
			MLP \cite{kucherenko2020gesticulator}, RNN \cite{bhattacharya2021speech2affectivegestures}, \cite{liu2022learning}, \cite{hasegawa2018evaluation}, \cite{yoon2020speech}, CNN \cite{habibie2021learning}, Transformer \cite{bhattacharya2021text2gestures}  & 
			- Khả năng ước lượng mật độ xác suất của dữ liệu \newline 
			- Có khả năng mở rộng và học từ dữ liệu lớn & 
			- Dễ bị ảnh hưởng bởi nhiễu \newline 
			- Kết quả thấp ở vùng dữ liệu hiếm \newline
			- Khả năng sinh không đa dạng & 
			Likelihood-Based Models \\ \hline
			\textbf{DiffusionStyle-Gesture} \cite{yang2022DiffuseStyleGestureplus}, MDM \cite{tevet2022human}, Motiondiffuse \cite{zhang2022motiondiffuse} &
			- Tạo ra dữ liệu chất lượng cao. \newline 
			- Linh hoạt và đa dạng \newline
			- Phủ được vùng có mật độ dữ liệu thấp & 
			- Cần cấu hình phức tạp để đạt hiệu năng tốt. \newline 
			- Khó đánh giá do mỗi lần nhiễu là khác nhau. \newline 
			- Quá trình lấy mẫu chậm & 
			Implicit Generative Models \\ \hline
		\end{tabular}
	}
	\caption{Bảng so sánh ưu và nhược điểm của các phương pháp}
\end{table}

%Trong mục tiêu theo luận văn sẽ trình bày các phương pháp sử dụng mô hình Diffusion.

%chuyển phân phối của dữ liệu ở dạng phân phối chuẩn hay ở một vị trí bất kỳ về phân phối chuẩn bằng hàm neuron network. 


%WGAN \cite{wu2021probabilistic}.



%\textbf{Bảng so sánh các phương pháp}
%
%\begin{table}[ht]
%	\centering
%	\begin{tabular}{|l|l|l|l|l|l|}
%		\hline
%		\textbf{Phương pháp} & \textbf{Loại mô hình} & \textbf{Đặc điểm nổi bật} & \textbf{Ưu điểm} & \textbf{Hạn chế} & \textbf{Tài liệu tham khảo} \\ \hline
%		VAE  & Autoencoder & Biểu diễn dữ liệu trong không gian tiềm ẩn & Tạo đặc trưng ẩn & Khó kiểm soát đầu ra & \cite{kingma2013auto} \\ \hline
%		VQ-VAE & Autoencoder (cải tiến) & Dùng codebook cho không gian tiềm ẩn & Biểu diễn chi tiết hơn & Phức tạp hơn & \cite{van2017neural} \\ \hline
%		RNN & Mạng hồi tiếp & Xử lý chuỗi dữ liệu & Tốt cho dữ liệu tuần tự & Khó huấn luyện & \cite{bhattacharya2021speech2affectivegestures} \\ \hline
%		Transformer & Mạng chú ý & Tạo cử chỉ qua cơ chế chú ý & Hiệu quả với dữ liệu dài & Yêu cầu nhiều tài nguyên & \cite{bhattacharya2021text2gestures} \\ \hline
%		WGAN & GAN & Học phân phối dữ liệu đối kháng & Tạo sự đa dạng & Khó huấn luyện & \cite{wu2021probabilistic} \\ \hline
%		Normalizing Flow & Mô hình xác suất & Học phân phối phức tạp & Hữu ích với dữ liệu phức tạp & Cần tài nguyên tính toán lớn & \cite{alexanderson2020style} \\ \hline
%		Diffusion Models & Mô hình sinh dữ liệu & Chi tiết cao, xử lý dữ liệu thiếu & Tạo cử chỉ chi tiết & Thời gian huấn luyện lâu & \cite{xu2022freeform} \\ \hline
%	\end{tabular}
%	\caption{So sánh các phương pháp sinh cử chỉ}
%\end{table}

%