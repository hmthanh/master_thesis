
\pagebreak
\phantomsection
\addcontentsline{toc}{section}{Trang thông tin luận văn tiếng Anh}
\begin{center}
	{\centering \MakeUppercase \LARGE \fontsize{16.16}{19.26}\selectfont \bfseries THESIS INFORMATION}
\end{center}



{
	\setlength{\parindent}{0pt}
Thesis title: OpenHuman: A conversational gesture synthesis system based on emotions and semantics \\
Speciality: Computer Science \\
Code: 8480104\\
Name of Master Student: Hoàng Minh Thanh \\
Academic year: 31\\
Supervisor: Associate Professor Ly Quoc Ngoc\\
At: VNUHCM - University of Science}


\vspace{5pt}
{\MakeUppercase\Large \bfseries{1. SUMMARY:}}

Along with the explosion of hardware advancements, large language models, text-based AI systems such as ChatGPT, CharacterAI, Gemini, and the development of computer graphics, the current bottleneck in creating digital humans lies in generating character movements corresponding to text or speech inputs.

One major challenge in gesture generation is the lack of sufficient and high-quality gesture data, as well as the lack of contextual consistency in gestures, making system development more difficult. Among current approaches, diffusion models achieve the best results due to their ability to generalize and handle imbalanced feature distributions and low-data-density regions effectively.

To generate gestures corresponding to speech and emotional information, this thesis employ a conditional diffusion model. The conditions include initial gestures, emotions, speech, and corresponding text. This thesis extends the DiffuseStyleGesture model to propose the \textbf{OHGesture} model. The thesis method converts speech into text as an additional semantic feature for training. We leverage the Unity codebase of the DeepPhase model for rendering and visualizing character movements. Lastly, this thesis make all our source code publicly available to encourage further development of gesture generation systems by the research community.


\vspace{5pt}
{\MakeUppercase\Large \bfseries 2. NOVELTY OF THESIS:}


%2. NOVEL CONTRIBUTIONS OF THE THESIS:
First, through actual gesture generation results, this thesis demonstrate that the proposed OHGesture model achieves high-quality gesture generation with synchronization between gestures, speech, and emotions.

Second, by integrating text to supplement semantic features, this thesis provide an additional mechanism to help the model understand specific contexts during gesture generation.

Third, this research contributes a novel method for constructing standardized evaluation systems for gesture generation models. By combining data from multiple linguistic sources and utilizing human evaluations, this thesis establish an objective and comprehensive benchmarking platform.

Finally, this thesis utilize animated character visualization, source code hosted on GitHub, and pretrained models on Huggingface. The proposed model achieves favorable results even with inputs outside the training dataset.

\vspace{5pt}
{\MakeUppercase\Large \bfseries 3. APPLICATIONS/ APPLICABILITY/ PERSPECTIVE:}

The OHGesture model is capable of generating gestures from emotion labels, speech, and corresponding text. This enables the development of human-interactive systems by integrating it with other text-based processing agents like Character.AI or the ChatGPT API. Additionally, we envision creating a store similar to the App Store but dedicated to virtual humans. Each virtual human, acting as an app, could be developed by various contributors and serve diverse applications such as virtual girlfriends, education, and customer assistance.

Currently, the model treats gesture sequences as images for denoising, which limits its performance. We plan to improve it by employing Fast Fourier Transform (FFT) to extract phase features of movements, aiming to develop a better gesture generation system.

\begin{center}
    \begin{tabular}{c c}
        \textbf{SUPERVISOR} & \textbf{Master STUDENT} \\
        (Signature, full name) & (Signature, full name) \\
    \end{tabular}
    
    \vspace{2.5cm} % Điều chỉnh khoảng cách này nếu cần
    
    \textbf{CERTIFICATION} \\
    \textbf{CERTIFICATION UNIVERSITY OF SCIENCE} \\
    \textbf{PRESIDENT}
\end{center}

\pagebreak