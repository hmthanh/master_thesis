
\pagebreak
\phantomsection
\addcontentsline{toc}{section}{Danh mục các ký hiệu, các chữ viết tắt}
\section*{\textbf{ \Large DANH MỤC CÁC KÝ HIỆU, CÁC CHỮ VIẾT TẮT}}

\begin{center}
\begin{tabular}{|p{3cm}|p{11cm}|}
\hline
\textbf{Từ viết tắt} & \textbf{Nội dung đầy đủ} \\
\hline
$\theta$ & Trọng số huấn luyện cần để học \\
\hline
$\theta'$ & Trọng số huấn luyện sau khi học xong dùng để lấy mẫu \\
\hline
$\epsilon$ & Nhiễu được thêm vào ảnh \\
\hline
$\hat{\epsilon}$ & Nhiễu dự đoán\\
\hline
$\epsilon_\theta$ hay $f_\theta$ & Hàm dự đoán nhiễu DDPM với trọng số $\theta$, $f_\theta$ và $\epsilon_\theta$ là một. \\
\hline
$N$, $M$, $D$ & Các số thể hiệu số chiều của ma trận hoặc vector \\
\hline
$t: 1 \rightarrow T$ & Bước gây nhiễu $t$ từ bước $t=1$ đến $t=T$ \\
\hline
$n_{\operatorname{joins}}$ & Số khung xương \\
\hline
$\mathcal{N}(a \bx, b^2)$ & Một hàm $f(x) = a \bx + b \epsilon$ với $\epsilon \in \mathcal{N}(0, \mathbf{1})$ thì tương đương phân phối chuẩn $\mathcal{N}(a \bx, b^2)$ \\
\hline
$\mathcal{N}(0, \mathbf{I})$ & Phân phối chuẩn với mean là $0$ và phương sai là $1$ \\
\hline
$\mathcal{N}(\bx_t; \mu_\theta, \Sigma_\theta)$ & Phân phối chuẩn với $\bx_t$ là output, $\mu$ trong bình của phân phối chuẩn đó với tham số $\theta$,  hàm dựa trên tham số $\theta$ có phương sai  $\Sigma$ \\ 
\hline
$\mathbf{x}_{t}^{1:M}$ & Dữ liệu cử chỉ ở bước thứ $t$, bắt đầu từ khung hình thứ $1$ đến khung hình thứ $M$ \\
\hline
$\hat{\mathbf{x}}_0$ &  Dữ liệu dự đoán của mô hình \\
\hline
$f(x | y)$ & Hàm xác suất có điều kiện với y có trước tìm xác suất để có x \\
\hline
$q(\bx_t | \bx_{t-1})$ & Hàm xác suất có điều kiện của hàm gây nhiễu, khi biết trước $\bx_t$ và tìm $\bx_{t-1}$  \\
\hline
$p_{\theta} (\bx_{t-1} | \bx_{t})$ & Hàm xác suất của quá trình khử nhiễu khi biết trước $\bx_{t}$ và tìm $\bx_{t-1}$ . Quá trình học để cập trọng số $\theta$   \\
\hline
$\boldsymbol{\mu}_\theta(\mathbf{x}_t, t)$ & Giá trung bình của $\bx_t$ ở bước thứ $t$ sau khi đi qua mô hình với trọng số $\theta$  \\
\hline
$\mathbb{E}$ & Kỳ vọng \\
\hline
$\mathcal{L}_t$ & Hàm mất mát ở bước thứ $t$ \\
\hline
$G_\theta$ & Mô hình OHGesture cần huấn luyện với trọng số $\theta$ \\
\hline
$\prod^T_{t=1}$ & Hàm tích số học, $\prod_{t=1}^T a_t = a_1 \cdot a_2 \cdot a_3 \cdot \ldots \cdot a_T$  \\
\hline


\end{tabular}
\end{center}


\pagebreak
