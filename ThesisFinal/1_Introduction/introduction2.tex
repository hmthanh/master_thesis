

\section{Mở đầu}
\label{sec:introduction}

\textit{Chương \ref{sec:introduction} đặt nền móng cho toàn bộ luận văn, bắt đầu bằng việc trình bày bối cảnh và lý do chọn đề tài. Đầu tiên, chương này phân tích các vấn đề hiện tại mà hệ thống quản lý cơ sở dữ liệu truyền thống như MSSQL gặp phải khi xử lý khối lượng dữ liệu lớn và nhu cầu mở rộng trong bối cảnh các hệ thống ngày càng phức tạp. Điều này bao gồm những hạn chế về khả năng mở rộng ngang, khả năng quản lý dữ liệu phân tán, và sự suy giảm hiệu suất khi phải xử lý lượng dữ liệu lớn.}

\textit{Tiếp theo, chương này giới thiệu mục tiêu chính của luận văn, đó là nghiên cứu và áp dụng các công nghệ cơ sở dữ liệu phân tán để cải thiện hiệu suất và khả năng mở rộng của hệ thống. Bên cạnh đó, chương này cũng xác định rõ phạm vi của nghiên cứu, giới hạn trong việc tập trung vào các công nghệ liên quan đến cơ sở dữ liệu phân tán mà không mở rộng sang các lĩnh vực khác. Điều này giúp xác định rõ ràng những gì sẽ được đề cập trong luận văn và những gì nằm ngoài phạm vi nghiên cứu.}


Trong thời đại kỹ thuật số hiện nay, dữ liệu đóng vai trò quan trọng trong hầu hết mọi khía cạnh của công nghệ thông tin. Đặc biệt, trong các hệ thống quản lý thành viên trực tuyến như ASP.NET Membership database \footnote{\url{https://learn.microsoft.com/en-us/previous-versions/aa479030(v=msdn.10)?redirectedfrom=MSDN}}, việc xử lý và quản lý dữ liệu lớn trở thành một thách thức đáng kể. MS SQL Server (MSSQL), một hệ quản trị cơ sở dữ liệu quan hệ phổ biến, đã được sử dụng rộng rãi trong việc quản lý dữ liệu này. Tuy nhiên, với sự phát triển nhanh chóng của dữ liệu và yêu cầu ngày càng cao về hiệu suất và khả năng mở rộng, MSSQL bắt đầu cho thấy những hạn chế của mình.

Khi đối mặt với lượng dữ liệu ngày càng tăng, MSSQL thường xuất hiện một số vấn đề nghiêm trọng ảnh hưởng đến hiệu suất và khả năng mở rộng. Đầu tiên và rõ ràng nhất là vấn đề tốc độ truy vấn chậm. Trong bối cảnh dữ liệu lớn, việc tìm kiếm và xử lý dữ liệu trở nên phức tạp và tốn thời gian, đặc biệt với các truy vấn liên kết nhiều bảng và có nhiều điều kiện. Điều này không chỉ gây trở ngại cho quá trình truy xuất dữ liệu mà còn làm giảm trải nghiệm người dùng cuối.

Một vấn đề khác là quản lý tài nguyên và bộ nhớ không hiệu quả. MSSQL có thể không tối ưu hóa việc sử dụng bộ nhớ và tài nguyên máy chủ khi phải đối phó với lượng lớn giao dịch và dữ liệu. Các vấn đề như sự phân mảnh (fragmentation) của dữ liệu và quản lý bộ nhớ cache không hiệu quả có thể xuất hiện, làm tăng độ trễ trong việc truy cập dữ liệu và ảnh hưởng đến hiệu suất tổng thể.

Ngoài ra, MSSQL cũng gặp phải hạn chế về khả năng mở rộng. Trong mô hình truyền thống, việc mở rộng quy mô cơ sở dữ liệu, đặc biệt là mở rộng ngang, không phải lúc nào cũng dễ dàng hoặc kinh tế. Việc tăng cường cơ sở hạ tầng có thể đòi hỏi đầu tư lớn và làm phức tạp quá trình quản lý dữ liệu, đặc biệt là trong các hệ thống cần xử lý lượng lớn dữ liệu liên tục và đồng thời.

Chính những hạn chế này đã tạo ra nhu cầu cấp thiết cho một giải pháp thay thế, và đây là lúc cơ sở dữ liệu phân tán bước vào. Với khả năng mở rộng linh hoạt, quản lý hiệu quả tài nguyên và bộ nhớ, cùng với việc giảm thiểu độ trễ trong xử lý truy vấn nhờ vào cơ chế phân phối dữ liệu qua nhiều node, cơ sở dữ liệu phân tán trở thành một giải pháp hấp dẫn và hiệu quả hơn trong việc xử lý các thách thức của dữ liệu lớn.

Trong bối cảnh này, cơ sở dữ liệu phân tán đưa ra một giải pháp hứa hẹn. Khác biệt so với mô hình truyền thống, cơ sở dữ liệu phân tán có thể phân chia và quản lý dữ liệu trên nhiều máy chủ, giúp giảm tải cho từng hệ thống và tăng cường hiệu suất tổng thể. Điều này không chỉ giúp giảm độ trễ trong truy vấn mà còn cung cấp khả năng mở rộng và bảo mật dữ liệu tốt hơn.

Mục tiêu của luận văn này là khám phá và phân tích sâu rộng về việc chuyển đổi từ MSSQL sang cơ sở dữ liệu phân tán, nhằm giải quyết các vấn đề liên quan đến quản lý dữ liệu lớn trong ASP.NET Membership. Bằng cách này, nghiên cứu nhằm đề xuất một giải pháp có thể cải thiện đáng kể hiệu suất, khả năng mở rộng, và độ tin cậy của hệ thống quản lý dữ liệu.

Để đảm bảo tính khả thi luận văn này sẽ không bao gồm các cải tiến về giao diện thành viên hoặc các tính năng không trực tiếp liên quan đến xử lý dữ liệu cốt lõi. Thay vào đó, luận văn sẽ tập trung vào việc cải thiện các quá trình kỹ thuật nhằm mục tiêu cải tiến trực tiếp hiệu suất và khả năng mở rộng của hệ thống.


Nghiên cứu này không chỉ có ý nghĩa với các nhà phát triển và quản trị viên hệ thống mà còn đóng góp vào lĩnh vực nghiên cứu cơ sở dữ liệu. Kết quả từ nghiên cứu này có thể được áp dụng rộng rãi trong các hệ thống quản lý dữ liệu lớn khác, đặc biệt trong các ứng dụng web hiện đại yêu cầu cao về hiệu suất và khả năng mở rộng.