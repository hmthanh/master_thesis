
\phantomsection
\addcontentsline{toc}{section}{Trang thông tin luận văn tiếng Việt}
\section*{\centering \MakeUppercase{TRANG THÔNG TIN LUẬN VĂN}}


Tên đề tài luận văn: OpenHuman: Hệ thống tổng hợp cử chỉ hội thoại đa phương thức dựa trên văn bản và âm thanh \\
Ngành: Khoa học máy tính \\
Mã số ngành:  Science \\
Họ tên học viên cao học: Hoàng Minh Thanh \\
Khóa đào tạo: 31 \\
Người hướng dẫn khoa học: PGS. TS. Lý Quốc Ngọc \\
Cơ sở đào tạo: Trường Đại học Khoa học Tự nhiên, ĐHQG.HCM


\subsection*{1 TÓM TẮT NỘI DUNG LUẬN VĂN}
Trong bối cảnh phát triển nhanh chóng của dữ liệu lớn và các hệ thống thông tin hiện đại, việc sử dụng Microsoft SQL Server (MSSQL) để quản lý các hệ thống thành viên trực tuyến như ASP.NET Membership đã bắt đầu bộc lộ nhiều hạn chế nghiêm trọng. ASP.NET Membership, vốn được thiết kế để quản lý thành viên và các thông tin liên quan, đã chứng kiến sự gia tăng đáng kể về khối lượng dữ liệu cần xử lý, đặc biệt là khi hệ thống ngày càng mở rộng. Tuy nhiên, khi đối mặt với khối lượng dữ liệu lớn và yêu cầu về tốc độ xử lý, MSSQL thường gặp phải các vấn đề về hiệu suất, như tốc độ truy vấn chậm, quản lý tài nguyên không hiệu quả, và khả năng mở rộng bị giới hạn.

Những hạn chế này không chỉ làm giảm trải nghiệm thành viên mà còn đặt ra thách thức lớn cho các nhà quản trị hệ thống trong việc duy trì và nâng cao hiệu suất của ASP.NET Membership. Để khắc phục các vấn đề này, luận văn đề xuất và triển khai một giải pháp chuyển đổi từ MSSQL sang hệ cơ sở dữ liệu phân tán Greenplum. Giải pháp này nhằm mục đích cải thiện đáng kể hiệu suất và khả năng mở rộng của ASP.NET Membership, đáp ứng nhu cầu xử lý dữ liệu lớn và yêu cầu ngày càng cao của các hệ thống quản lý thành viên trực tuyến.


Phần thực nghiệm của luận văn tiến hành các kiểm thử nghiêm ngặt để so sánh hiệu suất của hệ thống sau khi chuyển đổi sang Greenplum với hệ thống MSSQL ban đầu. Các bài kiểm tra tập trung vào những tác vụ quan trọng như tìm kiếm, đăng ký, và đăng nhập thành viên. Các chỉ số hiệu suất như độ trễ truy vấn, tốc độ xử lý và khả năng mở rộng đã được đo lường cẩn thận. Kết quả thực nghiệm cho thấy Greenplum mang lại sự cải thiện đáng kể về tốc độ xử lý và khả năng mở rộng, đáp ứng tốt các yêu cầu khắt khe của hệ thống dữ liệu lớn.

Kết luận của luận văn khẳng định rằng việc chuyển đổi từ MSSQL sang Greenplum là một giải pháp hiệu quả để nâng cao hiệu suất và khả năng mở rộng của hệ thống quản lý thành viên trực tuyến ASP.NET Membership. Bên cạnh đó, luận văn còn mở ra hướng nghiên cứu tiếp theo về việc tích hợp các công cụ phân tích dữ liệu nâng cao và tối ưu hóa kiến trúc hệ thống để đáp ứng nhu cầu ngày càng tăng của các ứng dụng web hiện đại.


\subsection*{2 NHỮNG KẾT QUẢ MỚI CỦA LUẬN VĂN}
Luận văn đã đạt được một số kết quả nổi bật trong lĩnh vực quản trị cơ sở dữ liệu phân tán, đặc biệt là việc chứng minh được khả năng cải thiện hiệu suất và khả năng mở rộng khi chuyển đổi từ MSSQL sang Greenplum trong hệ thống ASP.NET Membership. Kết quả thực nghiệm cho thấy, các cơ sở dữ liệu phân tán, đặc biệt là Greenplum, đã giảm đáng kể thời gian truy vấn và tăng cường hiệu suất xử lý dữ liệu lớn. Ngoài ra, luận văn cũng đề xuất một quy trình chuyển đổi hệ thống cơ sở dữ liệu tập trung sang hệ thống phân tán một cách hiệu quả, từ khâu chuẩn bị phần cứng, cài đặt phần mềm, đến tối ưu hóa cấu hình hệ thống.


\subsection*{3 CÁC ỨNG DỤNG/ KHẢ NĂNG ỨNG DỤNG TRONG THỰC TIỄN
HAY NHỮNG VẤN ĐỀ CÒN BỎ NGỎ CẦN TIẾP TỤC NGHIÊN CỨU}

Luận văn không chỉ có giá trị lý thuyết mà còn mở ra nhiều khả năng ứng dụng thực tiễn trong việc quản lý dữ liệu lớn, đặc biệt trong các hệ thống web yêu cầu cao về hiệu suất và tính sẵn sàng. Các doanh nghiệp có thể áp dụng giải pháp này để nâng cao hiệu suất hệ thống hiện có và đáp ứng nhu cầu mở rộng trong tương lai. Tuy nhiên, vẫn còn một số vấn đề cần tiếp tục nghiên cứu như tối ưu hóa thêm về khả năng chịu lỗi và an toàn dữ liệu trong các môi trường phân tán lớn hơn, cũng như tích hợp sâu hơn các công cụ phân tích dữ liệu và học máy để khai thác tối đa tiềm năng của hệ thống cơ sở dữ liệu phân tán.

\begin{center}
    \begin{tabular}{c c}
        \textbf{TẬP THỂ CÁN BỘ HƯỚNG DẪN} & \textbf{HỌC VIÊN CAO HỌC} \\
        (Ký tên, họ tên) & (Ký tên, họ tên) \\
    \end{tabular}
    
    \vspace{3cm} % Điều chỉnh khoảng cách này nếu cần
    
    \textbf{XÁC NHẬN CỦA CƠ SỞ ĐÀO TẠO} \\
    \textbf{HIỆU TRƯỞNG}
\end{center}
