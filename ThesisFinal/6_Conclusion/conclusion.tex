
\section{Kết luận}

\textit{Chương này tổng hợp lại những kết quả đạt được từ quá trình nghiên cứu và thực hiện luận văn, đồng thời đánh giá mức độ hoàn thành các mục tiêu đã đề ra ban đầu. Những thành tựu nổi bật bao gồm việc xây dựng thành công hệ thống cơ sở dữ liệu phân tán dựa trên Greenplum để thay thế MSSQL, cải thiện hiệu suất và khả năng mở rộng cho hệ thống ASP.NET Membership. Chương cũng trình bày bảng đánh giá mức độ hoàn thành công việc, liệt kê các công việc đã thực hiện và so sánh kết quả đạt được với mục tiêu đề ra. Ngoài ra, chương này còn đề xuất hướng phát triển tiếp theo, tập trung vào việc tối ưu hóa hệ thống và mở rộng khả năng ứng dụng trong các môi trường khác.}

\subsection{Các sản phẩm đạt được}

Luận văn đã hoàn thành một số sản phẩm quan trọng, góp phần nâng cao đáng kể hiệu suất và khả năng mở rộng của hệ thống quản lý thành viên trực tuyến, bao gồm:

\textbf{Thiết kế và triển khai hệ thống:} Luận văn đã trình bày một quy trình chi tiết để chuyển đổi cơ sở dữ liệu từ MSSQL sang Greenplum, bao gồm các bước quan trọng như ánh xạ dữ liệu, tích hợp với các provider của ASP.NET Membership. Quá trình này không chỉ đơn thuần là di chuyển dữ liệu mà còn yêu cầu sự hiểu biết sâu sắc về cả hai hệ thống, cũng như khả năng giải quyết các vấn đề tương thích. Đặc biệt, luận văn đã xử lý thành công các thách thức liên quan đến dữ liệu nhị phân và mã hóa mật khẩu, những yếu tố có tính phức tạp cao trong quá trình chuyển đổi.

\textbf{Tối ưu hóa hiệu suất:} Một trong những đóng góp nổi bật của luận văn là việc chứng minh hiệu quả của Greenplum trong việc nâng cao hiệu suất xử lý dữ liệu lớn. Các thử nghiệm thực tế đã cho thấy Greenplum không chỉ tốt hơn mà còn về khả năng xử lý đồng thời và độ ổn định. Điều này đặc biệt quan trọng trong các tác vụ phức tạp, nơi yêu cầu xử lý dữ liệu lớn và phục vụ nhiều thành viên đồng thời.

\textbf{Khả năng mở rộng:} luận văn cũng đã kiểm chứng khả năng mở rộng của Greenplum thông qua việc tăng số lượng node từ 3 lên 6. Kết quả cho thấy rằng khi hệ thống được mở rộng, không chỉ tốc độ xử lý được cải thiện đáng kể mà tính ổn định của hệ thống cũng được nâng cao, đáp ứng tốt hơn nhu cầu ngày càng tăng về xử lý dữ liệu lớn và số lượng thành viên đồng thời.

\textbf{Tài liệu và công cụ hỗ trợ:} Luận văn cung cấp một tài liệu chi tiết về toàn bộ quá trình nghiên cứu, bao gồm các phân tích kỹ thuật, so sánh hiệu suất và các giải pháp đã được áp dụng. Việc sử dụng hiệu quả các công cụ như PDI và Apache JMeter cũng được nhấn mạnh, góp phần tự động hóa và đơn giản hóa quy trình triển khai và đánh giá hệ thống, giúp người dùng có thể tiếp cận và áp dụng các phương pháp này một cách hiệu quả. Đồng thời, các script tùy chỉnh đã được phát triển để giải quyết các vấn đề cụ thể trong quá trình chuyển đổi và tích hợp.

\subsection{Khả năng áp dụng của đề tài vào thực tế}

Luận văn này không chỉ mang lại giá trị lý thuyết mà còn mở ra nhiều khả năng ứng dụng thực tiễn, đặc biệt trong việc cải thiện hiệu suất và khả năng mở rộng của hệ thống quản lý thành viên trực tuyến. Bằng việc chuyển đổi từ Microsoft SQL Server (MSSQL) sang Greenplum, một hệ thống cơ sở dữ liệu phân tán mạnh mẽ, các doanh nghiệp có thể nâng cao hiệu suất xử lý dữ liệu lớn và tăng cường khả năng mở rộng của hệ thống quản lý thành viên, như ASP.NET Membership.

Ứng dụng Greenplum giúp giảm đáng kể thời gian phản hồi và nâng cao trải nghiệm của người dùng, đồng thời hỗ trợ các tổ chức trong việc đáp ứng các yêu cầu ngày càng cao về tốc độ và tính linh hoạt của hệ thống. Hơn nữa, việc sử dụng Greenplum trong quản lý dữ liệu lớn không chỉ tối ưu hóa tài nguyên mà còn giảm thiểu chi phí, từ đó nâng cao khả năng cạnh tranh của doanh nghiệp trên thị trường.

Với các tính năng tiên tiến như xử lý song song và khả năng mở rộng linh hoạt, Greenplum không chỉ đáp ứng được nhu cầu hiện tại mà còn có khả năng thích ứng và mở rộng để phục vụ các nhu cầu trong tương lai. Điều này đặc biệt quan trọng trong bối cảnh dữ liệu lớn và yêu cầu phân tích ngày càng phức tạp của các doanh nghiệp hiện đại, nơi mà khả năng xử lý dữ liệu nhanh chóng và hiệu quả là yếu tố quan trọng.

\subsection{Bảng đánh giá mức độ hoàn thành công việc}

\begin{longtable}{|c|p{8cm}|p{8cm}|}
\hline
\textbf{STT} & \textbf{Mục tiêu đề ra ban đầu} & \textbf{Kết quả thu được} \\ \hline
\endfirsthead
\hline
\textbf{STT} & \textbf{Mục tiêu đề ra ban đầu} & \textbf{Kết quả thu được} \\ \hline
\endhead
\hline
\endfoot
\endlastfoot
1 & Tìm hiểu cơ sở dữ liệu phân tán & Đã nghiên cứu và áp dụng thành công các khái niệm và kỹ thuật liên quan đến cơ sở dữ liệu phân tán. \\ \hline
2 & Tăng cường hiệu suất truy vấn và giảm thời gian phản hồi & Hiệu suất truy vấn được cải thiện đáng kể, thời gian phản hồi giảm trung bình 10\% đến 30\% so với MSSQL. \\ \hline
3 & Đảm bảo khả năng mở rộng của hệ thống & Hệ thống có khả năng mở rộng ngang linh hoạt, thêm node mà không ảnh hưởng đến hoạt động hiện tại. \\ \hline
4 & Khám phá và triển khai quy trình ETL & Đã nghiên cứu và thực hiện thành công các quy trình ETL, áp dụng vào hệ thống để chuyển đổi và tích hợp dữ liệu. \\ \hline
5 & Tiến hành các thí nghiệm để đánh giá hiệu suất & Đã thực hiện các thí nghiệm đánh giá hiệu suất với kết quả phản ánh đúng thực tế và hỗ trợ cho quá trình phân tích. \\ \hline
6 & Cài đặt công cụ đánh giá hiệu suất & Đã cài đặt và cấu hình thành công các công cụ đánh giá hiệu suất, bao gồm cả JMeter. \\ \hline
7 & Viết báo cáo lại toàn bộ quá trình thực hiện luận văn & Báo cáo được viết đầy đủ, chi tiết, bao quát toàn bộ quá trình thực hiện luận văn từ nghiên cứu đến đánh giá. \\ \hline
\caption{Đánh giá mức độ hoàn thành các nhiệm vụ đề ra} \\
\end{longtable}



\subsection{Hướng phát triển}

Mặc dù luận văn đã đạt được những kết quả khả quan trong việc chuyển đổi và tối ưu hóa hệ thống quản lý thành viên ASP.NET Membership bằng Greenplum, nhưng vẫn còn nhiều hướng phát triển tiềm năng để nâng cao hơn nữa hiệu suất, tính ổn định và khả năng ứng dụng của hệ thống.

Một hướng quan trọng là tối ưu hóa hiệu suất truy vấn phức tạp. Điều này có thể đạt được bằng cách phát triển các thuật toán truy vấn mới, tận dụng tối đa khả năng xử lý song song của Greenplum, và cải tiến quy trình xử lý dữ liệu, bao gồm phân vùng và lập lịch tác vụ. Mục tiêu là đảm bảo hệ thống hoạt động hiệu quả ngay cả khi đối mặt với khối lượng dữ liệu khổng lồ và các yêu cầu truy vấn phức tạp.

Một hướng phát triển đầy hứa hẹn khác là tích hợp Greenplum với các công cụ phân tích dữ liệu tiên tiến như trí tuệ nhân tạo (AI) và học máy (Machine Learning). Sự kết hợp này không chỉ nâng cao khả năng xử lý dữ liệu của hệ thống mà còn mở ra tiềm năng phân tích và dự đoán dựa trên dữ liệu lớn. Bằng cách sử dụng AI và Machine Learning, hệ thống có thể cung cấp những phân tích sâu sắc hơn, nhận diện các xu hướng tiềm ẩn trong dữ liệu và hỗ trợ đưa ra quyết định chính xác hơn, giúp doanh nghiệp tối ưu hóa hoạt động và tăng cường năng lực cạnh tranh.

Ngoài ra, việc mở rộng khả năng ứng dụng của Greenplum cũng là một hướng đi quan trọng. Các kiến thức và kinh nghiệm thu được từ luận văn này có thể được áp dụng để chuyển đổi và tối ưu hóa các hệ thống quản lý dữ liệu khác, không chỉ giới hạn trong ASP.NET Membership.

